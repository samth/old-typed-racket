% quick.tex 
\documentclass{book} 
\usepackage{slatex} 

\author{Sam Tobin-Hochstadt}
\title{Typed Scheme: A Brief Introduction for PLT Scheme Programmers}
\date{\today}

\begin{document} 
\maketitle
\setspecialsymbol{planetpkg}{\scheme|("plt" "typed-scheme.plt")|}
\setspecialsymbol{tlang}{\scheme|(planet "typed-scheme.ss"  planetpkg)|}
\setspecialsymbol{readerlib}{\scheme|(planet "typed-reader.ss"  planetpkg)|}
\setkeyword{define: let: : lambda: pdefine: plambda: define-struct define-typed-struct module .. -> case-lambda pred All letrec: let*:  case-lambda: pcase-lambda: :: # #reader require}
\setkeyword{Un define-type-alias require/typed}

\newcommand{\tlang}{\scheme|tlang|}

\begin{schemeregion}




%\chapter{Introduction}

Typed Scheme is a Scheme-like language, with a type system that
supports common Scheme programming idioms.  Explicit type declarations
are required --- that is, there is no type inference.  The language
supports a number of features from previous work on type systems that
make it easier to type Scheme programs, as well as a novel idea dubbed
``occurrence typing'' for case discrimination.

Typed Scheme is also designed to integrate with the rest of your PLT
Scheme system.  It is possible to convert a single module to Typed
Scheme, while leaving the rest of the program unchanged.  The typed
module is protected from the untyped code base via
automatically-synthesized contracts.

Further information on Typed Scheme is available from
{\tt http://www.ccs.neu.edu/home/samth/typed-scheme.html} .

\subsection*{Installation}

There is no need to specifically install Typed Scheme.  Simply running
one of the example programs described in this manual will
automatically download and install the necessary software.  However,
it can also be installed from the MzScheme or DrScheme prompt with the
following require statement:

\begin{schemedisplay}
(require (planet "info.ss" planetpkg))
\end{schemedisplay}

In order to use the Typed Scheme language level in DrScheme, DrScheme
must be restarted after installation.

\tableofcontents

\chapter{Starting with Typed Scheme}

If you already know PLT Scheme, or even some other Scheme, it should be
easy to start using Typed Scheme.

\section{A First Function}

The following program defines the Fibonnaci function in PLT Scheme:

\begin{schemedisplay}
(module fib mzscheme
  (define (fib n)
    (cond [(= 0 n) 1]
          [(= 1 n) 1]
          [else (+ (fib (- n 1)) (fib (- n 2)))])))
\end{schemedisplay} 

This program defines the same program using Typed Scheme.
 
\begin{schemedisplay}
(module fib tlang
  (define: (fib [n : number]) : number
    (cond [(= 0 n) 1]
          [(= 1 n) 1]
          [else (+ (fib (- n 1)) (fib (- n 2)))])))
\end{schemedisplay} 

There are three differences between these programs:

\begin{itemize}
  \item {\bf The Language:} \scheme|mzscheme| has been replaced in the
    language position by \tlang{}.

  \item {\bf The Binding Form:} \scheme|define| has been replaced by
    \scheme|define:| as the top-level definition form.

  \item {\bf The Type Annotations:} Both the argument \scheme|n| and
    the result (appearing after the arguments) have been annotated
    with types, in this case \scheme|number|.
\end{itemize}


In general, these are most of the changes that have to be made to a PLT Scheme
program to transform it into a Typed Scheme program.\footnote{Changes
  to uses of \scheme|require| may also be necessary --- these are
  described later.}




\section{Adding more complexity}



Other typed binding forms are also available.  For example, we could have
rewritten our fibonacci program as follows:

\begin{schemedisplay}
(module fib tlang
  (define: (fib [n : number]) : number
    (let: ([base? : boolean (or (= 0 n) (= 1 n))])
      (if base? 1
          (+ (fib (- n 1)) (fib (- n 2))))))))
\end{schemedisplay} 

This program uses the \scheme|let:| binding form, which, like
\scheme|define:|, allows types to be provided, as well as the
\scheme|boolean| type.

We can also define mutually-recursive functions:

\begin{schemedisplay}
(module even-odd tlang
  (define: (my-odd? [n : number]) : boolean
    (if (= 0 n) #f
        (my-even? (- n 1))))

  (define: (my-even? [n : number]) : boolean
    (if (= 0 n) #t
        (my-odd? (- n 1))))

  (display (my-even? 12)))
\end{schemedisplay} 

As expected, this program prints \schemeresult|#t|.

\section{Defining New Datatypes}

If our program requires anything more than atomic data, we must define
new datatypes.  In Typed Scheme, structures can be defined, similarly
to PLT Scheme structures.  The following program defines a date
structure and a function that formats a date as a string, using PLT
Scheme's built-in \scheme|format| function.

%% too bad date and string are taken by mzscheme
\begin{schemedisplay}
(module date tlang

  (define-typed-struct my-date ([day : number] [month : str] [year : number]))
  
  (define: (format-date [d : my-date]) : str
    (format "Today is day ~a of ~a in the year ~a" (my-date-day d) (my-date-month d) (my-date-year d)))
  
  (display (format-date (make-my-date 28 "November" 2006)))
\end{schemedisplay}

Here we see the new built-in type \scheme|str| as well as a definition
of the new user-defined type \scheme|my-date|.  To define
\scheme|my-date|, we provide all the information usually found in a
\scheme|define-struct|, but added type annotations to the fields.
Then we can use the functions that this declaration creates, just as
we would have with \scheme|define-struct|.

\section{Recursive Datatypes and Unions}

Many data structures involve multiple variants.  In Typed Scheme, we
represent these using {\it union types}, written \scheme|(U t1 t2 ...)|

\begin{schemedisplay}
(module tree tlang
  (define-typed-struct leaf ([val : number]))
  (define-typed-struct node ([left : (U node leaf)] [right : (U node leaf)]))
  
  (define: (tree-height [t : (U node leaf)]) : number
    (cond [(leaf? t) 1]
          [else (max (tree-height (node-left t))
                     (tree-height (node-right t)))]))

  (define: (tree-sum [t : (U node leaf)]) : number
    (cond [(leaf? t) (leaf-val t)]
          [else (+ (tree-sum (node-left t))
                   (tree-sum (node-right t)))])))
\end{schemedisplay}

In this module, we have defined two new datatypes: \scheme|leaf| and
\scheme|node|.  We've also used the type \scheme|(U node leaf)|,
which represents a binary tree of numbers.  In essence, we are saying
that the \scheme|tree-height| function accepts either a \scheme|node|
or a \scheme|leaf| and produces a number.  

In order to calculate interesting facts about trees, we have to take
them apart and get at their contents.  But since accessors such as
\scheme|node-left| require a \scheme|node| as input, not a
\scheme|(U node leaf)|, we have to determine which kind of input we
were passed.  

For this purpose, we use the predicates that come with each defined
structure.  For example, the \scheme|leaf?| predicate distinguishes
\scheme|leaf|s from all other Typed Scheme values.  Therefore, in the
first branch of the \scheme|cond| clause in \scheme|tree-sum|, we know
that \scheme|t| is a \scheme|leaf|, and therefore we can get its value
with the \scheme|leaf-val| function.

In the else clauses of both functions, we know that \scheme|t| is not
a \scheme|leaf|, and since the type of \scheme|t| was \scheme|(U node
leaf)| by process of elimination we can determine that \scheme|t| must
be a \scheme|node|.  Therefore, we can use accessors such as
\scheme|node-left| and \scheme|node-right| with \scheme|t| as input.

\section{Giving Names to Types}

When a complex type is used repeatedly in a program, it can be helpful
to give it a short name.  In Typed Scheme, this can be done with {\it
  type aliases}.  For example, we could have used a type alias to
represent our tree datatype from the previous section.

\begin{schemedisplay}
(module tree tlang
  (define-typed-struct leaf ([val : number]))
  (define-typed-struct node ([left : (U node leaf)] [right : (U node leaf)]))
  
  (define-type-alias tree (U node leaf))
  
  (define: (tree-height [t : tree]) : number
    (cond [(leaf? t) 1]
          [else (max (tree-height (node-left t))
                     (tree-height (node-right t)))]))
  
  (define: (tree-sum [t : tree]) : number
    (cond [(leaf? t) (leaf-val t)]
          [else (+ (tree-sum (node-left t))
                   (tree-sum (node-right t)))])))
\end{schemedisplay}

The defintion \scheme|(define-type-alias tree (U node leaf))| creates
the type alias \scheme|tree|, which is just another name for the type
\scheme|(U node leaf)|.  We can then use this type in subsequent
definitions such as \scheme|tree-sum|.

\chapter{Polymorphism}

Virtually every Scheme program uses lists and sexpressions.  Fortunately, Typed
Scheme can handle these as well.  A simple list processing program can be
written like this:

\begin{schemedisplay}
(module add-list tlang
  (define: (sum-list [l : (Listof number)]) : number
    (cond [(null? l) 0]
          [else (+ (car l) (sum-list (cdr l)))])))
\end{schemedisplay}

This looks similar to our earlier programs --- except for the type
of \scheme|l|, which looks like a function application.  In fact, it's
a use of the {\it type constructor} \scheme|Listof|, which takes
another type as its input, here \scheme|number|.  We can use
\scheme|Listof| to construct the type of any kind of list we might
want.  

We can define our own type constructors as well.  For example, here is
an analog of the {\it Maybe} type constructor from Haskell:

\begin{schemedisplay}
(module maybe "typed-lang.ss"
  (define-typed-struct Nothing ())
  (define-typed-struct (a) Just ([v : a]))
  
  (define-type-alias (Maybe a) (U Nothing (Just a)))
  
  (define: (find [v : number] [l : (Listof number)]) : (Maybe number)
    (cond [(null? l) (make-Nothing)]
          [(= v (car l)) (make-Just v)]
          [else (find v (cdr l))])))
\end{schemedisplay}

The first \scheme|define-typed-struct| defines \scheme|Nothing| to be
a structure with no contents.  

The second definition
\begin{schemedisplay}
(define-typed-struct (a) Just ([v : a]))
\end{schemedisplay}
creates a parameterized type, \scheme|Just|, which is a structure with
one element, whose type is that of the type argument to
\scheme|Just|.  Here the type parameters (only one, \scheme|a|, in
this case) are written before the type name, and can be referred to in
the types of the fields.

The type alias definiton
\begin{schemedisplay}
  (define-type-alias (Maybe a) (U Nothing (Just a)))
\end{schemedisplay}
creates a parameterized alias --- \scheme|Maybe| is a potential
container for whatever type is supplied.

The \scheme|find| function takes a number \scheme|v| and list, and
produces \scheme|(make-Just v)| when the number is found in the list,
and \scheme|(make-Nothing)| otherwise.  Therefore, it produces a
\scheme|(Maybe number)|, just as the annotation specified.  

\chapter{More Built-In Types}

Typed Scheme comes with a number of other types and type constructors,
corresponding (mostly) to primitive types provided by PLT Scheme.

\renewcommand{\i}[2]{\item #1 #2}

\begin{itemize}
\item The base types represent primitive data
  \begin{itemize}
    \i{\scheme|number|}{Any number}
    \i{\scheme|str|}{A string}
    \i{\scheme|symbol|}{A symbol}
    \i{\scheme|boolean|}{Either \scheme|#t| or \scheme|#f|}
    \i{\scheme|keyword|}{A PLT Scheme literal keyword}
    \i{\scheme|top|}{Any value}
  \end{itemize}
\item The following constructors are parameteric in a single type argument
  \begin{itemize}
    \i{\scheme|(Listof t)|}{Homogenous lists of \scheme|t|}
    \i{\scheme|(Vectorof t)|}{Homogenous vectors of \scheme|t|}
    \i{\scheme|(Option t)|}{Either \scheme|t| or \scheme|#f|}
%    \i{\scheme|(box-of t)|}{A singleton container of \scheme|t|}
  \end{itemize}
\i{\scheme|(cons s t)|}{is the pair containing \scheme|s| as the \scheme|car|
  and \scheme|t| as the \scheme|cdr|}
\i{\scheme|(dom ... -> rng)|}{is the type of functions from the (possibly-empty)
  sequence \scheme|dom ...| to the \scheme|rng| type.}
\i{\scheme|(dom ... rst .. -> rng)|}{is the type of functions from the
  (possibly-empty) sequence \scheme|dom ...| with an optional trailing
  sequence of \scheme|rst| to the \scheme|rng| type. Note: \scheme|..| is a
  part of the syntax of these types.}
\i{\scheme|(U t ...)|}{is the union of the types \scheme|t ...|}
\i{\scheme|(case-lambda fun-ty ...)|}{is a function that behaves like all of
  the \scheme|fun-ty|s.  The \scheme|fun-ty|s must all be function
  types constructed with \scheme|->|.}
\i{\scheme|(t t1 t2 ...)|}{is the instantiation of the parametric type
  \scheme|t| at types \scheme|t1 t2 ...|}
\i{\scheme|(pred t)|}{is a predicate for type \scheme|t|}
\i{\scheme|(All (v ...) t)|}{is a parameterization of type \scheme|t|, with
  type variables \scheme|v ...|}
\i{\scheme|(values t ...)|}{is the type of a sequence of multiple values, with
types \scheme|t ...|.  This can only appear as the return type of a
function.}
\i{\scheme|v|}{where \scheme|v| is a number, boolean or string, is the singleton type containing
only that value}
\i{\scheme|i|}{where \scheme|i| is an identifier can be a reference to a type
name or a type variable}
\end{itemize}

Other types cannot be written by the programmer, but are used
internally and may appear in error messages.

%\scheme|n| is a name, \scheme|t| is a type.

\begin{itemize}
        \i{\scheme|#<struct:n (t ...)>|}{is the type of structures named
\scheme|n| with field types \scheme|t|.  There may be multiple such
types with the same printed representation.}
        \i{\scheme|(mu n t)|}{is a recursive type where \scheme|n| is bound to the
recursive type in the body \scheme|t|}
        \i{\scheme|<n>|}{is the printed representation of a reference to the
type variable \scheme|n|}
\end{itemize}


\chapter{Special Form Reference}

Typed Scheme provides a variety of special forms above and beyond
those in PLT Scheme.  They are used for annotating variables with types,
creating new types, and annotating expressions.

\section{Binding Forms}

\scheme|loop|, \scheme|f|, \scheme|a|, and \scheme|v| are names, \scheme|t| is a type.
 \scheme|e| and \scheme|body| are expressions.

\begin{itemize}
  \i{\scheme|(define: (f [v : t] ...) : t body)|}{}
  \i{\scheme|(define: v : t e)|}{}
  \i{\scheme|(pdefine: (a ...) (f [v : t] ...) : t body)|}{}
  \i{\scheme|(let: ([v : t e] ...) body)|}{}
  \i{\scheme|(let: loop : t0 ([v : t e] ...) body)|}{where \scheme|t0| is the type of the
  result of \scheme|loop| (and thus the result of the entire expression).}
  \i{\scheme|(letrec: ([v : t e] ...) body|}{}
  \i{\scheme|(let*: ([v : t e] ...) body)|}{}
  \i{\scheme|(lambda: ([v : t] ...) body)|}{}
  \i{\scheme|(lambda: ([v : t] ... . [v : t]) body)|}{}
  \i{\scheme|(plambda: (a ...) ([v : t] ...) body)|}{}
  \i{\scheme|(plambda: (a ...) ([v : t] ... . [v : t]) body)|}{}
  \i{\scheme|(case-lambda: [formals body] ...)|}{where \scheme|formals| is like
  the second element of a \scheme|lambda:|}
  \i{\scheme|(pcase-lambda: (a ...) [formals body] ...)|}{where \scheme|formals| is like
  the third element of a \scheme|plambda:|}
\end{itemize}

\section{Structure Defintions}

\scheme|name|, \scheme|parent|, \scheme|f|, and \scheme|v| are names,
\scheme|t| is a type.  \scheme|parent| must name a structure type.

\begin{itemize}
  \i{\scheme|(define-typed-struct name ([f : t] ...))|}{}
  \i{\scheme|(define-typed-struct (name parent) ([f : t] ...))|}{}
  \i{\scheme|(define-typed-struct (v ...) name ([f : t] ...))|}{}
  \i{\scheme|(define-typed-struct (v ...) (name parent) ([f : t] ...))|}{}
\end{itemize}

\section{Type Aliases}

\scheme|name| and \scheme|v| are names, \scheme|t| is a type.

\begin{itemize}
  \i{\scheme|(define-type-alias name t)|}{}
  \i{\scheme|(define-type-alias (name v ...) t)|}{}
\end{itemize}

\section{Type Annotation}

\scheme|v| is a variable name, \scheme|e| is an expression, \scheme|t| is a type.

These annotations require the use of the following declaration before
the beginning of the module:

\begin{schemedisplay}
#reader readerlib
\end{schemedisplay}

This will not affect the reading or parsing of any other syntax.

\begin{itemize}
  \i{\scheme|#{v : t}|}{This is legal only for binding occurences of \scheme|v|}
  \i{\scheme|#{e :: t}|}{This is legal only in expression contexts}
\end{itemize}

\section{Require}

\scheme|v| is a variable name, \scheme|t| is a type, and \scheme|m| is
a require-spec.

\begin{itemize}
        \i{\scheme|(require/typed v t m)|}{}
\end{itemize}

\end{schemeregion}

\chapter{The \textrm{Typed Scheme} Language Level}

In addition to providing a language to be used with PLT Scheme
modules, Typed Scheme is also available as a language selection within
DrScheme.  To enable this language, simply choose it in the ``Choose
Language'' dialog found in the ``Language'' menu of DrScheme.  Typed
Scheme programs can then be entered in the definitions window, and
executed with the run button.  

Unlike some other languages available in DrScheme, the Typed Scheme
language level does not simulate a top-level read-eval-print loop.
Therefore, only one definition of a variable is allowed and uses of
unbound variables result in errors.

\end{document} 
