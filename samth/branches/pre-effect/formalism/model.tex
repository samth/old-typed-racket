\documentclass{article}[12pt]
\usepackage{mmm}
\usepackage{sample}
\usepackage{amsmath}
\usepackage{amsthm}
\usepackage{fullpage}
\usepackage{bcprules}
\newtheorem{theorem}{Theorem}
\newtheorem{lemma}{Lemma}
\newtheorem{define}{Definition}

\newcommand\red{\rightarrow^*} 
\newcommand{\sembrack}[1]{[\![#1]\!]}

\begin{document}

\title{Typed Scheme}
\author{Sam Tobin-Hochstadt}
\date{\today}

\newmeta\e{e}
\newmeta\E{E}
\newmeta\d{d}
\newmeta\b{b}
\newmeta\uu{u}
\newmeta\vv{v}
\newmeta\kk{k}
\newmeta\n{n}
\newmeta\ph{\phi}
%\newcommand\v{\vv{}}

\maketitle

\section{Syntax}

\[
%\begin{tdisplay}{Syntax of Expressions, Contexts and Types}
  \begin{altgrammar}
    \ph{} & ::=& \t{} \alt \noeffect & \mbox{Latent Lexical Effects} \\
    \p{} &::=& \t{\x{}} \alt \x{} \alt \tt \alt \ff \alt \noeffect & \mbox{Lexical Effects} \\
   \s{},\t{} &::= & \top \alt \num \alt \bool \alt {\proctype {\s{}} {\t{}} {\ph{}}} \alt (\usym\ \t{} \dots) &\mbox{Types} \\
   \d{}, \e{}, \dots &::=& \x{}  \alt \comb{\e1}{\e2} \alt \cond{\e1}{\e2}{\e3} \alt \vv{} \alt \wrong &\mbox{Expressions} \\
   \vv{} &::=& \c{}  \alt \b{} \alt \n{} \alt \abs{\x{}}{\t{}}{\e{}}  & \mbox{Values} \\
   \c{} &::=& \addone \alt \numberp \alt \boolp & \mbox{Constants} \\
   \E{} &::=& [] \alt \comb{\E{}}{\e{}} \alt \comb{\vv{}}{\E{}} \alt \cond{\E{}}{\e2}{\e3} & \mbox{Evaluation Contexts}
   \end{altgrammar}
%\end{tdisplay}
\]

Note that $(\usym)$ is below all other types, and that \wrong is a normal form but not a value.

\section{Static Semantics}


%\subsection{Typing Rules}

\judgment{Expression Typing}{\ghastyeff{\e{}}{\t{}}{\p{}}}

\[
\inferrule[T-Var]{}{\ghastyeff{\x{}}{\G{}(\x{})}{\x{}}}
\qquad\qquad
\inferrule[T-Num]{}{\ghasty{\n{}}{\num}}
\qquad\qquad
\inferrule[T-Const]{}{\G{} \vdash \hastype{\c{}}{\dt{\c{}}}}
\]

\[
\inferrule[T-True]{}{\ghastyeff{\tt}{\bool}{\tt}}
\qquad\qquad
\inferrule[T-False]{}{\ghastyeff{\ff}{\bool}{\ff}}
\]

\[
\inferrule[T-Abs]{\hastyeff{\G{},\hastype{\x{}}{\s{}}}{\e{}}{\t{}}{\p{}}}
        {\ghasty{\abs{\x{}}{\s{}}{\e{}}}{\proctype {\s{}} {\t{}} {\noeffect}}}
\qquad\qquad
\inferrule[T-App]{\ghastyeff{\e1}{\proctype {\t0} {\t1} {\ph{}}}{\p{}}
\\
\ghastyeff{\e2}{\t{}}{\p{}'} \\\\ \subtype{\t{}}{\t0}}
{\ghastyeff{\comb{\e1}{\e2}} {\t1} {\noeffect}}
\]

\[
\inferrule[T-If]
        {\ghastyeff{\e1}{\t1}{\p1}
        \\\\
        \hastyeff{\G{} + \p1}{\e2}{\t2}{\p2}
        \\\\
        \hastyeff{\G{} - \p1}{\e3}{\t3}{\p3}
        \\\\
        \subtype{\t2}{\t{}} \\ \subtype{\t3}{\t{}}
        }
        {\ghasty{\cond{\e1}{\e2}{\e3}}{\t{}}}
\qquad\qquad
\inferrule[T-AppPred]{\ghastyeff{\e1}{\proctype {\t0} {\t1} {\s{}}}{\p{}}
\\
\ghastyeff{\e2}{\t{}}{\x{}} \\\\ \subtype{\t{}}{\t0}}
{\ghastyeff{\comb{\e1}{\e2}} {\t1} {\s{\x{}}}}
\]


Note that these rules {\it require} that predicates always are
predicates for the {\it whole} type, that is, prime? is not a
predicate for \num.

%% should we have \subtype{\bool}{\t1} as a premise of these rules?
\[
\inferrule[T-AppPredTrue]{\ghastyeff{\e1}{\proctype{\t0}{\t1}{\s{}}}{\p{}} \\ \ghastyeff{\e2}{\t{}}{\p{}'}
  \\\\ \subtype{\t{}}{\t0} \\ \subtype{\t{}}{\s{}}}
        {\ghastyeff{\comb{\e1}{\e2}}{\t1}{\tt}}
\qquad\qquad
\inferrule[T-AppPredFalse]{\ghastyeff{\e1}{\proctype{\t0}{\t1}{\s{}}}{\p{}} \\ \ghastyeff{\e2}{\t{}}{\p{}'}
  \\\\ \subtype{\t{}}{\t0} \\ \t{} \cap \s{} = (\usym)}
        {\ghastyeff{\comb{\e1}{\e2}}{\t1}{\ff}}
\]

These next two rules are sort-of analgous to stupid casts.
Also, should they have less conservative effects?

\[
\inferrule[T-IfTrue]
        {\ghastyeff{\e1}{\t1}{\tt}
        \\
        \hastyeff{\G{}}{\e2}{\t2}{\p2}
        \\\\
        \subtype{\t2}{\t{}}
        }
        {\ghasty{\cond{\e1}{\e2}{\e3}}{\t{}}}
\qquad\qquad
\inferrule[T-IfFalse]
        {\ghastyeff{\e1}{\t1}{\ff}
        \\
        \hastyeff{\G{}}{\e3}{\t3}{\p3}
        \\\\
        \subtype{\t3}{\t{}}
        }
        {\ghasty{\cond{\e1}{\e2}{\e3}}{\t{}}}
\]


%\infrule[T-Sub]
%        {\subtype{\t{}}{\s{}} \andalso \ghastyeff{\e{}}{\t{}}{\p{}}}
%        {\ghastyeff{\e{}}{\s{}}{\p{}}}


\judgment{Subtyping Rules}{\subtype{\s{}}{\t{}}}

\[
\inferrule[S-Refl]{}{\subtype{\t{}}{\t{}}}
\qquad\qquad
\inferrule[S-Trans]{\subtype{\t1}{\t2} \\ \subtype{\t2}{\t3}}
        {\subtype{\t1}{\t3}}
\qquad\qquad
\inferrule[S-Fun]{\subtype{\t1}{\s1} \\ \subtype{\s2}{\t2}}
        {\subtype{\proctype{\t1}{\t2}{\p{}}}{\proctype{\s1}{\s2}{\p{}}}}
\]

\[
\inferrule[S-UnionSuper]{\subtype{\t{}}{\s{i}} \\ 1 \leq i \leq n}
        {\subtype{\t{}}{(\usym\ \s1 \cdots \s{n})}}
\qquad\qquad
\inferrule[S-UnionSub]{\subtype{\t{i}}{\s{}} \mbox{\ for all $1 \leq i \leq n$}}
        {\subtype{(\usym\ \t1 \cdots \t{n})}{\s{}}}
\]

\subsection{Auxilliary Operations}

\judgment{Environment Update}{\G{} \pm \p{} = \G{}}

$$
\G{} + \t{\x{}} = \G{}[\hastype{\x{}}{\G{}(x) \cap \t{}}]
\qquad
\G{} + \x{} = \G{}
\qquad
\G{} + \noeffect = \G{}
$$
$$
\G{} - \t{\x{}} = \G{}[\hastype{\x{}}{\G{}(x) - \t{}}]
\qquad
\G{} - \x{} = \G{}
\qquad
\G{} - \noeffect = \G{}
$$

\judgment{Constant Typing}{\dt{\c{}} = {\t{}}}

$$
\dt{\addone} = \proctype{\num}{\num}{\noeffect} ; \noeffect
$$
$$
\dt{\numberp} = \proctype{\top}{\bool}{\num} ; \noeffect
$$
$$
\dt{\boolp} = \proctype{\top}{\bool}{\bool} ; \noeffect
$$

\section{Operational Semantics}

\judgment{Reduction Rules}{\reduce{\d{}}{\e{}}}

For these rules, $\E{}[]$ is implicitly wrapped around both sides.

\[
\inferrule*[lab={E-Delta}]
        {\del{\c{}}{\vv{}} = \vv{}'}
        {\reduce{\comb{\c{}}{\vv{}}}{\vv{}'}}
\qquad\qquad
\inferrule[E-Beta]{}
      {\reduce{\comb{\abs{\x{}}{\t{}}{\e1}}{\e2}}{\subs{\e1}{\x{}}{\e2}}}
\]

\[
\inferrule[E-IfFalse]{}
      {\reduce{\cond{\ff}{\e2}{\e3}}{\e3}}
\qquad\qquad
\inferrule[E-IfTrue]{}
      {\reduce{\cond{\vv{}}{\e2}{\e3}}{\e2}}
\]

Here the context is explicit.

\[
\inferrule[E-DeltaWrong]
        {\del{\c{}}{\vv{}} = \wrong}
        {\reduce{\E{}[\comb{\c{}}{\vv{}}]}{\wrong}}
\qquad\qquad
\inferrule[E-AppWrong]              %
        {\vv1 \mbox{\ is not an abstraction}}
        {\reduce{\E{}[\comb{\vv1}{\vv2}]}{\wrong}}
\]

\judgment{Constant Operations}{\del{\c{}}{\vv{}} = \vv{}}


In these definitions, matching is left-to-right.

$$
\del{\addone}{\n{}} = \n{} + 1
\qquad
\del{\addone}{\vv{}} = \wrong
$$
$$
\del{\numberp}{\n{}} = \tt
\qquad
\del{\numberp}{\vv{}} = \ff
$$
$$
\del{\boolp}{\b{}} = \tt
\qquad
\del{\boolp}{\vv{}} = \ff
$$



\section{Soundness}

Here's the type soundness theorem we would like to have:

\begin{theorem}[Soundness]
  If \hastyeff{\ }{\e{}}{\t{}}{\p{}}  then one of the following holds:
  \begin{enumerate}
    \item \e{} reduces forever
    %\item \reduces{\e{}}{\wrong}
    \item \reduces{\e{}}{\vv{}} where \hastyeff{\ }{\vv{}}{\s{}}{\p{}'} and  \subtype{\s{}}{\t{}}.
  \end{enumerate}
\end{theorem}


Let's consider progress:
\begin{theorem}[Progress]
  If \ghastyeff{\e{}}{\t{}}{\p{}} then either \reduce{\e{}}{\wrong} or \reduce{\e{}}{\e{}'} for some $\e{}'$.
\end{theorem}

This seems fine. 


Here's the problem:  the following theorem is not true without the \tt and \ff effects.

\begin{theorem}[Subject Reduction]
  If \ghastyeff{\e{}}{\t{}}{\p{}}, and \reduce{\e{}}{\e{}'}, then
  \ghastyeff{\e{}'}{\t{}'}{\p{}'} where \subtype{\t{}'}{\t{}}.
\end{theorem}


{\bf Counter-Example:} 


Let \e0 be
\comb
    {\abs{\x{}}{(\usym \ \num \ \bool)}
      {\cond
        {\comb{\numberp}{\x{}}}
        {\comb{\addone}{\x{}}}
        {0}
      }}
    {\tt}.


Then \reduce{\e0}{      {\cond
        {\comb{\numberp}{\tt}}
        {\comb{\addone}{\tt}}
        {0}
      }}. 


But \comb{\addone}{\tt} is not well-typed, so the \ifsym expression is
not well-typed unless we avoid looking at the then branch somehow.



\newpage

\section{Possible New Rules}

These rules are possibilites that use the lexical effects to type more programs.

This rule applies when we $\eta$-expand a predicate, as in \abs{\x{}}{\t{}}{\comb{\numberp}{\x{}}}

\[
\inferrule[T-AbsPred]{\hastyeff{\G{},\hastype{\x{}}{\t0}}{\e{}}{\t{}}{\s{\x{}}}}
        {\ghasty{\abs{\x{}}{\t0}{\e{}}}{\proctype {\t0} {\t{}} {\s{}}}}
\]

This rule allows us to type the expansion of $(and\ \cdots)$

\[
%% named so because it handles the expansion of (and ...)
\inferrule[T-IfAnd]
        {\ghastyeff{\e1}{\t1}{\p1}
        \\\\
        \hastyeff{\G{} + \p1}{\e2}{\t2}{\p2}
        \\\\
        \hastyeff{\G{} - \p1}{\e3}{\t3}{\ff}
        \\\\
        \subtype{\t2}{\t{}} \\ \subtype{\t3}{\t{}}
        }
        {\ghastyeff{\cond{\e1}{\e2}{\e3}}{\t{}}{\p1}}
\]

provided that we change the returned effect in rules {\sc T-IfTrue}
and {\sc T-IfFalse} is made more specific - in particular \p2 and \p3
respectively.



\end{document}

